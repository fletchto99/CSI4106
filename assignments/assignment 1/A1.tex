\documentclass[fleqn, 12pt]{article}

% packages %
\usepackage[includeheadfoot,headheight=15pt,margin=0.5in,left=1in,right=1in,headsep=10pt]{geometry} % page margins %
\usepackage{mathtools, amssymb} % math %
\usepackage{tabularx, multirow} % tables %
\usepackage{minted} % code %
\usepackage{graphicx} % graphics %
\usepackage{enumerate} % lists %
\usepackage{adjustbox} % images %
\usepackage[T1]{fontenc} % fonts %
\usepackage[protrusion=true,expansion=true]{microtype} % font clarity %
\usepackage{fancyhdr} % headers and footers %
\usepackage{lastpage} % reference page numbers %
\usepackage{color} % colors for code %
\usepackage{tikz} % for graphs %
\usepackage{soul} % for strikethroughout %
\usepackage{upquote} % Fix single quotes %
\usepackage{etoolbox} % Conditional checks %
\usepackage{hyperref} % Hyperlinks %
\usepackage{indentfirst} % fix indentation - only for essays %
\usepackage[figure,table]{totalcount} % For counting tables and figures %
\usepackage[utf8]{inputenc} % Encode as UTF-8 %
\usepackage{biblatex} % References %
\addbibresource{references.bib} % bib source %

% Document details %
\newcommand{\university}{University of Ottawa}
\newcommand{\name}{Matt Langlois}
\newcommand{\studentNumber}{7731813}
\newcommand{\nameTwo}{Alexandre Billard}
\newcommand{\studentNumberTwo}{6812210}
\newcommand{\semester}{Fall 2017}
\newcommand{\assignmentType}{Assignment}
\newcommand{\assignmentNumber}{1}
\newcommand{\dueDate}{Oct. 10}
\newcommand{\courseCode}{CSI4106}
\newcommand{\courseTitle}{Intro to AI}
\newcommand{\essayTitle}{<Title>} % only for essays %
\newcommand{\essaySubtitle}{<subtitle>} % only for essays %
\newcommand{\essayAbstract}{} % Only for essays -- leave empty for no abstract %

% Center image and diagrams %
\adjustboxset*{center}

% Code settings %
\setminted{
    fontfamily=tt,
    gobble=0,
    frame=single,
    funcnamehighlighting=true,
    tabsize=4,
    obeytabs=false,
    mathescape=false
    samepage=false,
    showspaces=false,
    showtabs =false,
    texcl=false,
    breaklines=true,
}

% inline code %
\definecolor{codegray}{gray}{0.9}
\newcommand{\code}[1]{\colorbox{codegray}{\texttt{#1}}}

% Code from tile %
\newcommand{\codefile}{\inputminted}

% Graphing stuff %
\usetikzlibrary{arrows.meta}
\usetikzlibrary{positioning}
\usetikzlibrary{matrix}
\usetikzlibrary{automata}

% Define ceiling and floor functions %
\DeclarePairedDelimiter\ceil{\lceil}{\rceil}
\DeclarePairedDelimiter\floor{\lfloor}{\rfloor}

% Create set compliment command %
\newcommand{\setcomp}[1]{{#1}^{\mathsf{c}}}

% Create logic command aliases %
\newcommand{\limplies}{\rightarrow}
\newcommand{\nequiv}{\not\equiv}
\newcommand{\liff}{\leftrightarrow}

% first page header & footer %
\fancypagestyle{assignment}{
    \fancyhf{}
    \renewcommand{\footrulewidth}{0.1mm}
    \fancyfoot[R]{\assignmentType \assignmentNumber}
    \fancyfoot[C]{\thepage \hspace{1pt} of \pageref{LastPage}}
    \fancyfoot[L]{\courseCode}
    \renewcommand{\headrulewidth}{0mm}
}

% Frontmatter for essay page numbering%
\fancypagestyle{frontmatter}{
    \fancyhf{}
    \renewcommand{\footrulewidth}{0.1mm}
    \fancyfoot[R]{\assignmentType \assignmentNumber}
    \fancyfoot[C]{\thepage \hspace{1pt} of \pageref{EndFrontMatter}}
    \fancyfoot[L]{\courseCode}
    \fancyhead[L]{\name}
    \fancyhead[R]{\studentNumber}
}

% Essay body page numbering %
\fancypagestyle{body}{
    \fancyhf{}
    \renewcommand{\footrulewidth}{0.1mm}
    \fancyfoot[R]{\assignmentType \assignmentNumber}
    \fancyfoot[C]{\thepage \hspace{1pt} of \pageref{LastPage}}
    \fancyfoot[L]{\courseCode}
    \fancyhead[L]{\name}
    \fancyhead[R]{\studentNumber}
}

% Page header and footers %
\fancyhf{}
\renewcommand{\footrulewidth}{0.1mm}
\fancyfoot[R]{\assignmentType \assignmentNumber}
\fancyfoot[C]{\thepage \hspace{1pt} of \pageref{LastPage}}
\fancyfoot[L]{\courseCode}
\fancyhead[L]{\name\text{ - }\studentNumber}
\fancyhead[R]{\nameTwo\text{ - }\studentNumberTwo}

% Apply headers & footer page style %
\pagestyle{fancy}

% Assignment first page header %
\newcommand{\beginassignemnt}{
    % Prevent paragraph indents, this isn't an English assignment! %
    \newlength\tindent
    \setlength{\tindent}{\parindent}
    \setlength{\parindent}{0pt}

    \thispagestyle{assignment}
    \noindent
    \courseTitle \hfill \university\\
    \courseCode \hfill \semester
    \begin{center}
        \textbf{\assignmentType\text{ }\#\assignmentNumber}\\
        \name \hspace{1pt} - \studentNumber\\
        \nameTwo \hspace{1pt} - \studentNumberTwo\\
        \dueDate\\
    \end{center}
    \vspace{6pt}
    \hrule
    \vspace{1.5\headsep}
}

% Essay titlepage stuff %
\newcommand{\beginessay}{
    % Load all citations %
    \nocite{*}

    % Special numbering for front matter %
    \pagestyle{frontmatter}
    \pagenumbering{roman}

    % Titlepage stuff %
    \begin{center}
        \normalsize
        \textsc{\university}\\[5cm]
        \LARGE \textbf{\MakeUppercase{\essayTitle}}\\[0.5cm]
        \large \text{ }\essaySubtitle\text{ }\\[10cm] % blank \texts are used for empty subtitles %
        \normalsize
        \textsc{\name}\\
        \textsc{\studentNumber}\\
        \textsc{\courseCode}\\
        \textsc{\semester}\\
        \textsc{\dueDate}
    \end{center}
    \thispagestyle{empty}
    % End title page stuff %

    % Table of Contents %
    \newpage
    \tableofcontents
    \newpage

    % If more than 1 table/figure show appropriate lists %
    \iftotalfigures
        \addcontentsline{toc}{section}{\listfigurename}
        \listoffigures
    \fi
    \iftotaltables
        \addcontentsline{toc}{section}{\listtablename}
        \listoftables
    \fi

    % Display an abstract if the abstract isn't empty %
    \ifdefempty{\essayAbstract}{}{
        \newpage
        \addcontentsline{toc}{section}{Abstract}
        \begin{abstract}
            \essayAbstract
        \end{abstract}

    }
    \label{EndFrontMatter}
    \newpage

    % Reset page numbering %
    \pagenumbering{arabic}
    \pagestyle{body}
}

% Update the bibliography command to add its self to the references
\let\oldprintbib\printbibliography
\renewcommand{\printbibliography}{
    \newpage
    \oldprintbib
    \addcontentsline{toc}{section}{References}
    \newpage
}

% Fixes a Pygments bug in ES6 -- Thanks ShareLatex! %
\makeatletter
\expandafter\def\csname PYGdefault@tok@err\endcsname{\def\PYGdefault@bc##1{{\strut ##1}}}
\makeatother

% Begin the document %
\begin{document}

% makes the header for assignment/titlepage for essay %
% \beginessay
\beginassignemnt


\section*{Question 1}

For our assignment we used the specified version of Python. No specific environment variables were set and all testing was done via the command line using Python3. The output from \code{python --version} is \code{Python 3.6.2}. All testing was done on Mac OS X with python 3 installed via homebrew.

\section{Question 2}

\subsection*{Dataset 1 (EIGHT\_PUZZLE\_DATA[1])}

\begin{tabular}{|c|c|c|c|c|c|}
    \hline
        Algo & Depth & Cost & Nodes Visited & Time & Completes  \\\hline
        BFS  & 1 & 1 & 8 & 0.005 seconds & Yes \\\hline
        DFS  & 1 & 1 & 4 & 0.002 seconds & Yes \\\hline
        A* (H1)  & 1 & 1 & 2 & 0.002 seconds & Yes \\\hline
        A* (H2)  & 1 & 1 & 2 & 0.003 seconds & Yes \\
    \hline
\end{tabular}

\subsection*{Dataset 2 (EIGHT\_PUZZLE\_DATA[6])}

\begin{tabular}{|c|c|c|c|c|c|}
    \hline
        Algo & Depth & Cost & Nodes Visited & Time & Completes  \\\hline
        BFS  & 14 & 14 & 3264 & 0.244 seconds & Yes \\\hline
        DFS  & 63304 & 63304 & 131426 & 9.411 seconds & Yes \\\hline
        A* (H1)  & 14 & 14 & 284 & 0.164 seconds & Yes \\\hline
        A* (H2)  & 14 & 14 & 59 & 0.015 seconds & Yes \\
    \hline
\end{tabular}

\subsection*{Dataset 3 (EIGHT\_PUZZLE\_DATA[3])}

\begin{tabular}{|c|c|c|c|c|c|}
    \hline
        Algo & Depth & Cost & Nodes Visited & Time & Completes  \\\hline
        BFS  & 19 & 19 & 29926 & 2.388 seconds & Yes \\\hline
        DFS  & 39449 & 39449 & 167747 & 13.818 seconds & Yes \\\hline
        A* (H1)  & 19 & 19 & 2700 & 9.136 seconds & Yes \\\hline
        A* (H2)  & 19 & 19 & 292 & 0.185 seconds & Yes \\
    \hline
\end{tabular}

\section*{Question 3}

\begin{enumerate}[a)]
    \item
        Heuristic Descriptions:
        \begin{enumerate}[1)]
            \item
                Heuristic h1 computes the number of tiles which are in the wrong location. It does not compute how far they are from the correct position thus it can severely underestimate the true distance to the proper location.
            \item
                Heuristic h2 computes the Manhattan distance to the proper location. This is a more accurate estimate compared to h1 heuristic since it is computing not only if it is in the wrong position but how far of from the correct position the tile is.
        \end{enumerate}
    \item
        Admissibility:
        \begin{enumerate}[1)]
            \item
                H1 will never overestimate how many moves are required. Worst case it will only estimate how many tiles there are, if all tiles are moved. Thus it cannot overestimate.
            \item
                H2 is admissible because it estimates the distance every tile needs to travel to get back to its original position. This is the bare minimum a tile will need to move, this h2 will never over estimate and is admissable.
        \end{enumerate}
    \item
        Heuristic H2 dominates in all cases. We believe this is because H2 is a more accurate representation of how far a tile will need to move without over estimating the distance. H1 just checks if its in the wrong location stating it needs to move which essentially. This means the maximum value it will ever be for this question is 8, which can be way off compared to the total distance all tiles will need to move. This is why H2 is a much stronger approach.
\end{enumerate}

\section*{Question 4}

Given the best possible admissible heuristic then A* appears to be the best approach. Through our tests it appears that A* using the Manhattan heuristic was the best approach for this case. It was also noticed that DFS would sometimes generate extremely long sets of moves to get to the solution. For example in Dataset 3 DFS required 39449 moves while A* only required 19 moves. Overall A* appears to be the best algroithm for this task.

\end{document}
